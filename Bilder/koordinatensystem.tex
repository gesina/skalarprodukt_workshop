% Beispiel für ein Koordinatensystem mit einem Beispielvektor

\begin{tikzpicture}
  \begin{axis}[
    width=\linewidth,
    font=\scriptsize,
    axis equal, axis lines=center,
    xticklabels=\empty, yticklabels=\empty,
    xmin=-2.5, xmax=2.5, xtick distance=1,% minor x tick num=1,
    ymin=-2.5, ymax=2.5, ytick distance=1,% minor y tick num=1
    grid=major
    ]
    \coordinate (0) at (axis cs:0,0);
    \draw[fill] (0) circle[radius=2pt];
    \draw[thick, -Stealth] (0) -- (axis cs:1,0)
    node [pos=1, below] {$(1,0)$};
    \draw[thick, -Stealth] (0) -- (axis cs:0,1)
    node [pos=1, left] {$(0,1)$};
    
    \draw[-Stealth] (axis cs:1,0) -- (axis cs:2,0);
    \draw[-Stealth] (axis cs:2,0) -- (axis cs:2,1);
    % \draw[-Stealth] (axis cs:2,1) -- (axis cs:2,2);

    \coordinate (ex) at (axis direction cs:2,1);
    \coordinate (other) at (axis cs:-2.5,-1.5);
    \draw[thick, -Stealth] (0) -- ($(0)+(ex)$)
    node [pos=1, left] {$(2,1)$};
    \draw[thick, -Stealth] (other)  -- ($(other)+(ex)$)
    node [pos=1, left] {$(2,1)$};
  \end{axis}
\end{tikzpicture}
